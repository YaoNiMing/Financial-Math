\documentclass[a4paper]{article}
\usepackage{amsmath}
\usepackage{amsthm}
\newtheorem{thm}{Theorem}
\usepackage[english]{babel}
\usepackage[utf8]{inputenc}
\usepackage{graphicx}
\usepackage[colorinlistoftodos]{todonotes}
\usepackage[margin=1in]{geometry}
\usepackage{tikz}
\usetikzlibrary{trees}
\usetikzlibrary{matrix}


\title{The Binomial No-Arbitrage Pricing Model}

\author{Peiliang Guo}

\date{\today}
\begin{document}
\maketitle

\section{One-Period Binomial Model}
Stock price per share at $t_0$: $S_0$. At $t_1$, stock price will be one of $S_1(H)=uS_0$ and $S_1(T)=dS_0$, depending on the result of a toss of a coin. Result is measurable at $t_1$ and unknown at $t_0$. We do not need $p=\mathbf{P}(H)$ to equal $q=1-p=\mathbf{P}(T)$. 

\begin{center}

  \begin{tikzpicture}[>=stealth,sloped]
    \matrix (tree) [%
      matrix of nodes,
      minimum size=1cm,
      column sep=3.5cm,
      row sep=1cm,
    ]
    {
            & $S_1(H) = uS_0$    \\
      $S_0$ &    \\
          & $S_1(T)=dS_0$    \\
    };
    \draw[->] (tree-2-1) -- (tree-1-2) node [midway,above] {$p$};
    \draw[->] (tree-2-1) -- (tree-3-2) node [midway,below] {$q=1-p$};
      \end{tikzpicture}\end{center}
Assume $u>r$. Also introduce interest rate $r$ such that $1$ dollar at $t_0$ becomes $1+r$ dollars at $t_1$, only $r>-1$ is required. With the assumption that the stock price is always positive, to avoid arbitrage, we requrie $$0<d<1+r<u$$  
Consider a \textit{European Call option}, which gives its owner the right but not the obligation to buy one share of stock at the \textit{strike price} $K$. i.e. payoff$=(S_1-K)^+$. \\
The \textit{arbitrage pricing theory} replicates the option by trading the stock and money markets. The no-arbitrage price of the option should be the time-zero price of the portfolio of stocks and money markets that replicates the time-one payoff of the option. It assumes the following:
\begin{itemize}
\item share of stock can be subdivided for sale or purchase
\item the interest rate for investing is the same as the interest rate for borrowing
\item the purchase price of stock is the same as the selling price
\item at any time, the stock can take only two possible values in the next period
\end{itemize}
In general, suppose a derivative security pays $V_1(H)$ and $V_1(T)$ at $t_1$, and suppose we can replicate the portfolio with $\Delta_0$ shares of stock and $X_0-\Delta_0S_0$ in the money market. Then our $t_1$ portfolio value is 
$$X_1 = \Delta S_1+(1+r)(X_0-\Delta S_0) = (1+r)X_0+\Delta_0(S_1-(1+r)S_0)$$
We need $X_1(H) = V_1(H)$ and $X_1(T) = V_1(T)$, i.e.
\begin{align*}&X_0 + \Delta_0\left(\frac{1}{1+r}S_1(H) - S_0\right)=\frac{1}{1+r}V_1(H)\\
&X_0+\Delta_0\left(\frac{1}{1+r}S_1(T)-S_0\right) = \frac{1}{1+r}V_1(T)\end{align*}
Introduce artificial quantity $\tilde{p}$ and $\tilde{q}=1-\tilde{p}$, such that $$S_0 = \frac{1}{1+r}\left[\tilde{p}S_1(H)+\tilde{q}S_1(T)\right]\Longrightarrow \tilde{p} = \frac{1+r-d}{u-d}, \tilde{q} = \frac{u-1-r}{u-d}$$
Then multiply the first equation by $\tilde{p}$ plus the second equation multiplied by $\tilde{q}$, we get 
$$X_0 =\frac{1}{1+r}[\tilde{p}V_1(H)+\tilde{q}V_1(T)]$$
Therefore, by the no-arbitrage argument, the derivative that pays $V_1$ at $t_1$ should be priced at $$V_0=\frac{1}{1+r}[\tilde{p}V_1(H)+\tilde{q}V_1(T)]$$ at $t_0$. 
$\tilde{p}$ and $\tilde{q}$ (together constitutes probability measure $\mathbf{Q}$) are artificial probabilities that makes \textbf{all traded assets} to have expected discount $r$. $\mathbf{Q}$ is called risk-neutral measure, since if $\mathbf{Q}$ is the probability, investors are risk neutral.
\section{Multiperiod Binomial Model}
Extend the single period model to multi-period.
\begin{center}
\begin{tikzpicture}[>=stealth,sloped]
    \matrix (tree) [%
      matrix of nodes,
      minimum size=1cm,
      column sep=3.5cm,
      row sep=1cm,
    ]
    {
    	& & $S_2(HH)=uuS_0$\\
            & $S_1(H) = uS_0$&    \\
      $S_0$ &   &$S_2(HT)=S_2(TH)=udS_0$ \\
          & $S_1(T)=dS_0$   & \\
          &&$S_2(TT)=ddS_0$\\
    };
    \draw[->] (tree-3-1) -- (tree-2-2) node [midway,above] {$p$};
    \draw[->] (tree-3-1) -- (tree-4-2) node [midway,below] {$q=1-p$};
        \draw[->] (tree-2-2) -- (tree-1-3) node [midway,above] {$p$};
    \draw[->] (tree-2-2) -- (tree-3-3) node [midway,below] {$q=1-p$};
        \draw[->] (tree-4-2) -- (tree-3-3) node [midway,above] {$p$};
    \draw[->] (tree-4-2) -- (tree-5-3) node [midway,below] {$q=1-p$};
      \end{tikzpicture}\end{center}
Suppose an European call option that pays $V_2 = (S_2-K)^+$, where $V_2$ and $S_2$ depend on the first and second coin toss. Suppose an agent sells the option for $V_0$ dollars, then she buys $\Delta_0$ shares of stock and invests $V_0-\Delta_0 S_0$ in the money market to hedge her position. At $t_1$, the agent has a portfolio (not including one unit short in the option) valued at 
\begin{align*}
X_1(H) = \Delta_0 S_1(H) + (1+r)(V_0-\Delta_0S_0)\\
X_1(T) = \Delta_0 S_1(T) + (1+r)(V_0-\Delta_0S_0)
\end{align*}
She readjust her positions to $\Delta_1$ shares of stock, and invests $X_1-\Delta_1S_1$ in the money market, then at $t_2$, she has
\begin{align*}
V_2(HH) &= \Delta_1(H)S_2(HH)+(1+r)(X_1(H)-\Delta_1(H)S_1(H))\\
V_2(HT) &= \Delta_1(H)S_2(HT)+(1+r)(X_1(H)-\Delta_1(H)S_1(H))\\
V_2(TH) &= \Delta_1(T)S_2(TH)+(1+r)(X_1(T)-\Delta_1(T)S_1(T))\\
V_2(TT) &= \Delta_1(T)S_2(TT)+(1+r)(X_1(T)-\Delta_1(T)S_1(T))
\end{align*}
There are six equations, with six unknowns: $V_0,X_1(H),X_1(T),\Delta_0,\Delta_1(H),\Delta_1(T)$. The solution can be solved by breaking the problem into three 1-period models
\begin{align*}
\Delta_1(T) &= \frac{V_2(TH)-V_2(TT)}{S_2(TH)-S_2(TT)}\\
V_1(T) &= X_1(T) = \frac{1}{1+r}[\tilde{p}V_2(TH)+\tilde{q}V_2(TT)]\\
\Delta_1(H) &= \frac{V_2(HH)-V_2(HT)}{S_2(HH)-S_2(HT)}\\
V_1(H) &= X_1(H) = \frac{1}{1+r}[\tilde{p}V_2(HH)+\tilde{q}V_2(HT)]\\
\Delta_0 &= \frac{V_1(H)-V_1(T)}{(S_1(H)-S_1(T)}\\
V_0 &= X_0 = \frac{1}{1+r}[\tilde{p}V_1(H) + \tilde{q}V_1(T)]
\end{align*}
$V_n$ is call the no-arbitrage price of the derivative security at time $n$. 
\begin{thm}\textbf{(Replication in the multiperiod binomial model)}
Consider an $N$-period binomial asset-pricing model, with $0<d<1+r<u$, and with
$$\tilde{p} = \frac{1+r-d}{u-d}, \tilde{q} = \frac{u-1-r}{u-d}$$
Let $V_N$ be a random variable (a derivative security paying off at time $N$) depending on the first $N$ coin tosses $\omega_1\omega_2...\omega_N$. Define recursively backward in time the sequence of random variables $V_{N-1},V_{N-2},...,V_0$ by 
$$V_n(\omega_1\omega_2...\omega_n)=\frac{1}{1+r}[\tilde{p}V_{n+1}(\omega_1\omega_2...\omega_nH)+\tilde{q}V_{n+1}(\omega_1\omega_2...\omega_nT)],$$ so that each $V_n$ depends on the first $n$ coin tosses $\omega_1\omega_2...\omega_n$ where $n$ ranges between $N-1$ and $0$. Next define
$$\Delta_n(\omega_1...\omega_n)=\frac{V_{n+1}(\omega_1...\omega_nH)-V_{n+1}(\omega_1...\omega_nT)}{S_{n+1}(\omega_1...\omega_nH)-S_{n+1}(\omega_1...\omega_nT)}$$
where again $n$ ranges between $0$ and $N-1$. If we set $X_0=V_0$ and define recursively forward in time the portfolio values $X_1,...,X_N$ by 
$$X_{n+1}=\Delta_nS_{n+1}+(1+r)(X_n-\Delta_n S_n)$$
then we will have 
$$X_N(\omega_1...\omega_N) = V_N(\omega_1...\omega_N) \,\text{ for all }\omega_1...\omega_N.$$
\end{thm}
Proof by forward induction.
Theorem also applies to path-dependent options. refer to example 1.2.4 in Shreve.

\end{document}
              