\documentclass[a4paper]{article}
\usepackage{amsmath,amsfonts}
\usepackage{amsthm}
\newtheorem{thm}{Theorem}
\theoremstyle{definition}
\newtheorem{defn}[thm]{Definition} 
\usepackage[english]{babel}
\usepackage[utf8]{inputenc}
\usepackage{graphicx}
\usepackage[colorinlistoftodos]{todonotes}
\usepackage[margin=1in]{geometry}
\usepackage{tikz}
\usetikzlibrary{trees}
\usetikzlibrary{matrix}


\title{State Prices}

\author{Peiliang Guo}

\date{\today}
\begin{document}
\maketitle
\section{Change of Measure}
\textit{actual probability measure} - the model which we seek by empirical estimation of the model parameters. \\
\textit{risk-neutral probability measure} - under which the discounted prices of assets are martingales. \\
The two measures agree on the state space (equivalent measures), but disagree on the probability of each state occurring. \\
Consider a finite sample space $\Omega$ on which we have two probability measures $\mathbb{P}$ and $\tilde{\mathbb{P}}$, both giving positive probability for every $\omega\in\Omega$. Form quotient 
$$Z(\omega) = \frac{\tilde{\mathbb{P}}(\omega)}{\mathbb{P}(\omega)}$$
$Z$ is a random variable because it depends on the outcome $\omega$ of a random experiment. $Z$ is called the \textit{Radon-Nikodym derivative of $\tilde{\mathbb{P}}$ w.r.t. $\mathbb{P}$}.
\begin{thm}
Let $\mathbb{P}$ and $\tilde{\mathbb{P}}$ be probability measures on finite sample space $\Omega$, assume that $\mathbb{P}(\omega)>0$ and $\tilde{\mathbb{P}}(\omega)>0$ for every $\omega\in\Omega$, and define random variable $Z$ as $Z(\omega) = \frac{\tilde{\mathbb{P}}(\omega)}{\mathbb{P}(\omega)}$. Then we have the following:\\
(i) $\mathbb{P}(Z>0) = 1$;\\
(ii) $\mathbb{E}Z = 1$;\\
(iii) for any random variable $Y$, 
$$\tilde{\mathbb{E}}Y = \mathbb{E}[ZY].$$
\end{thm} 
\begin{proof}
Property (i) follows immediately from the fact that we have assumed $\mathbb{P}(\omega)>0$ and $\tilde{\mathbb{P}}(\omega)>0$ for every $\omega \in \Omega$. \\
Property (ii) 
$$\mathbb{E}Z = \sum_{\omega\in\Omega}Z(\omega)\mathbb{P}(\omega) = \sum_{\omega\in\Omega} \frac{\tilde{\mathbb{P}}(\omega)}{\mathbb{P}(\omega)}\mathbb{P}(\omega)  = \sum_{\omega\in\Omega}\tilde{\mathbb{P}}(\omega) = 1$$
Property (iii)
$$\tilde{\mathbb{E}}Y = \sum_{\omega\in\Omega}Y(\omega)\tilde{\mathbb{P}}(\omega) = \sum_{\omega\in\Omega}Y(\omega)\frac{\tilde{\mathbb{P}}(\omega)}{\mathbb{P}(\omega)}\mathbb{P}(\omega) = \sum_{\omega\in\Omega}Z(\omega)Y(\omega)\mathbb{P}(\omega) = \mathbb{E}[ZY]$$
\end{proof}
\begin{defn}
In the $N$-period binomial model with actual probability measure $\mathbb{P}$ and risk-neutral probability measure $\tilde{\mathbb{P}}$, let $Z$ denote the Radon-Nikodym derivative of $\tilde{\mathbb{P}}$ w.r.t. $\mathbb{P}$; i.e. 
$$Z(\omega_1...\omega_N) = \frac{\tilde{\mathbb{P}}(\omega_1...\omega_N)}{\mathbb{P}(\omega_1...\omega_N)}=\left(\frac{\tilde{p}}{p}\right)^{\#H(\omega_1...\omega_N)}\left(\frac{\tilde{q}}{q}\right)^{\#T(\omega_1...\omega_N)}$$
The \textit{state price density} random variable is 
$$\zeta(\omega) = \frac{Z(\omega)}{(1+r)^N}$$
and $\zeta(\omega)\mathbb{P}(\omega)$ is called the \textit{state price} corresponding to $\omega$. 
\end{defn}
Note that the state price $\zeta(\bar{\omega})\mathbb{P}(\bar{\omega})$ is the price at time zero of a contract that pays $1$ at time $N$ if and only if $\bar{\omega}$ occurs. \\
For contracts that with payoffs not necessarily $1$, and with payoffs associated with more than one $\omega$, we can consider the contract as a portfolio of the "simple contracts", each of which pays off $1$ if and only if some particular $\omega$ occurs. To see this, note that 
$$V_0 = \tilde{\mathbb{E}}\frac{V_N}{(1+r)^N} = \mathbb{E}[\zeta V_N] = \sum_{\omega\in\Omega}V_N(\omega)\zeta(\omega)\mathbb{P}(\omega)$$
\section{Radon-Nikodym Derivative Process}
To estimate $Z$ based on information at time $n<N$, where $Z$ is not required to be a Radon-Kikodym derivative, we develop the following theorem
\begin{thm}
	Let $Z$ be a random variable in an $N$-period binomial model. Define
	$$Z_n = \mathbb{E}_n Z, n=0,1,...,N.$$
	Then $Z_n, n=0,1,...,N$ is a martingale under $\mathbb{P}$.
\end{thm}
\begin{proof}
	For $n=0,1,...,N-1$, we use the iterated conditioning property. 
	$$\mathbb{E}_n[Z_{n+1}]=\mathbb{E}_n[\mathbb{E}_{n+1}[Z]]=\mathbb{E}_n[Z]=Z_n$$
\end{proof}
Note that the martingale property holds not only under $\mathbb{P}$ but the risk-neutral measure $\tilde{\mathbb{P}}$ as well.
\begin{defn}
	In an $N$-period binomial model, let $\mathbb{P}$ be the actual probability measure, $\tilde{\mathbb{P}}$ the risk-neutral probability measure, and assume that $\mathbb{P}(\omega)>0$ and $\tilde{\mathbb{P}}(\omega)>0$ for every sequence of coin tosses $\omega$. Define the Randon-Nikodym derivative (random variable) $Z(\omega) = \frac{\tilde{\mathbb{P}}(\omega)>0}{\mathbb{P}(\omega)>0}$ for every $\omega$. The Radon-Nikodym derivative process is 
	$$Z_n = \mathbb{E}_n[Z], n=0,1,...,N$$
	In particular, $Z_N=Z$ and $Z_0=1$. 
\end{defn}
\begin{thm}
	Assume the conditions of above definition. Let $n$ be a positive integer between $0$ and $N$, and let $Y$ be a random variable depending only on the first $n$ coin tosses. Then 
	$$\tilde{\mathbb{E}}Y = \mathbb{E}[Z_n Y]$$
\end{thm}
\begin{proof}
	$$\tilde{\mathbb{E}}Y = \mathbb{E}[ZY]=\mathbb{E}[\mathbb{E}_n[ZY]]=\mathbb{E}[Y\mathbb{E}_n[Z]]=\mathbb{E}[YZ_n]$$
\end{proof}
\begin{thm}
	Consider an $N$-period binomial model with $0<d<1+r<u$. Let $\mathbb{P}$ and $\tilde{\mathbb{P}}$ denote the corresponding actual and risk-neutral probability measures, respectively. Let $Z$ be the Radon-Nikodym derivative of $\tilde{\mathbb{P}}$ w.r.t. $\mathbb{P}$, and let $Z_n,n=0,1,...,N$, be the Radon-Nikodym derivative process. 
	Consider a derivative security whose payoff $V_N$ may depend on all $N$ coin tosses. For $n=0,1,..,N$, the price at time $n$ of the derivative security is 
	$$V_n = \tilde{\mathbb{E}}_n\frac{V_N}{(1+r)^{N-n}}=\frac{(1+r)^n}{Z_n}\mathbb{E}_n\frac{Z_NV_N}{(1+r)^N}=\frac{1}{\zeta_n}\mathbb{E}_n[\zeta_NV_N],$$
	where the state price density process $\zeta_n$ is defined by 
	$$\zeta_n=\frac{Z_n}{(1+r)^n}, n=0,1,...,N.$$
\end{thm}
\section{Capital Asset Pricing Model}
\subsubsection*{Optimal Investment Problem}
Given $X_0$, find an adaped portfolio process $\Delta_0, \Delta_1,...,\Delta_{N-1}$ that maximizes
$$\mathbb{E}U(X_N)$$ subject to the wealth equation
$$X_{n+1} = \Delta_nS_{n+1}+(1+r)(X_n-\Delta_nS_n), n=0,1,...,N-1.$$
Note that the agent uses the acutal probability measure $\mathbb{P}$ and uses uitility function $U$ to capture the trade-off between risk and return. Using risk neutral probability does not make sense because under the risk-neutral measure, every asset will have the same expected return; an agent seeking to maximize $\tilde{\mathbb{E}}U(X)$ would invest only in the money market. 
However, as the number of periods increases, the number of varialbes $\Delta_n(\omega)$ grows exponentially. So we can also formulate the problem as follows:
\subsubsection*{Same Investment Problem Revisited}
Given $X_0$, find a random variable $X_N$ (without regard to a portfolio process) that maximizes
$$\mathbb{E}U(X_N)$$
subject to $$\tilde{\mathbb{E}}\frac{X_N}{(1+r)^N}=X_0$$
If we use superscripts to indicate that $\omega^m$ isa fulll sequence of coin tosses, and define $\zeta_m = \zeta(\omega^m)$, $p_m = \mathbb{P}(\omega^m)$, and $x_m = X_N(\omega^m)$. Then the problem can be reformulated as follows:\\
Given $X_0$, find a vector $(x_1,x_2,...,x_M)$ that maximizes 
$$\sum_{m=1}^Mp_m U(x_m)$$
subject to $$\sum_{m=1}^Mp_mx_m\zeta_m = X_0.$$
The Lagrange multiplier equations are 
$$\frac{\partial}{\partial x_m}L  = p_mU'(x_m) - \lambda p_m\zeta_m = 0, m=1,2,...,M$$
So$$ U'(x_m) = \lambda \zeta_m \Longrightarrow U'(X_N) = \frac{\lambda Z}{(1+r)^N}\Longrightarrow X_N = (U')^{-1}\left(\frac{\lambda Z}{(1+r)^N}\right)$$
Therefore, we can solve for $\lambda$ by 
$$X_0 = \mathbb{E}\left[\frac{Z}{(1+r)^N}(U')^{-1}\left(\frac{\lambda Z}{(1+r)^N}\right)\right]$$
\end{document}
              